% !TeX encoding=utf8
% !TeX spellcheck = de_CH_frami

%%% --- Acronym definitions
\IfDefined{newacronym}{%
% examples

% our used acronyms
% zhaw internal
\newacronym{zhawLabel}{ZHAW}{Zürcher Hochschule für Angewandte Wissenschaften}
\newacronym{ebsLabel}{EBS}{Einschreibe und Bewertungssystem der ZHAW}

% it general
\newacronym{macLabel}{MAC}{Media Access Control}
\newacronym{asciiLabel}{ASCII}{American Standard Code for Information Interchange}
\newacronym{httpsLabel}{HTTPS}{Hypertext Transfer Protocol Secure}

\newacronym{ieeeLabel}{IEEE}{Institute of Electrical and Electronics Engineers}
\newacronym{usbLabel}{USB}{Universal Serial Bus}
\newacronym{nfcLabel}{NFC}{Near Field Communication}


%network
\newacronym{arpLabel}{ARP}{Address Resolution Protocol}
\newacronym{rtsLabel}{RTS}{Ready to send}
\newacronym{ctsLabel}{CTS}{Clear to send}

% bluetooth basics




}

%%% --- Symbol list entries

%\newglossaryentry{symb:Pi}{%
%  name=$\pi$,%
%  description={mathematical constant},%
%  sort=symbolpi, type=symbolslist%
%}


%%% --- Glossary entries

\newglossaryentry{glos:rainbowTableLabel}{
	name={Rainbow Table},
	plural={Rainbow Tables},
	description={\textit{rainbow tables} beinhalten fertig berechnete Hashes um Passwörter schneller knacken zu können. Um die Nutzung der \textit{rainbow tables} einzuschränken, wird oft ein Salt in den Hash eingerechnet, so dass allgemein verfügbare \textit{rainbow tables} nicht verwendet werden können.}
}






%Glossary entries used also as acronyms
%\newglossaryentry{ssidLabel}{
%	name={SSID},
%	description={Mit der \textit{Service Set ID (SSID)} wird ein gesamtes Netzwerk identifiziert. Die \textit{SSID} entspricht dem sichtbaren Netzwerknamen und kann sich über mehrere \gls{apLabel}'s erstrecken.\footcite{Understanding_the_Network_Terms_SSID_BSSID_and_ESSID_Technical_Documentation_Support_2015-04-10}},
%	first={Service Set ID (SSID)},
%	long={Service Set ID}
%}
%
%\newglossaryentry{bssidLabel}{
%	name={BSSID},
%	description={Mit der \textit{Basis Service Set ID (BSSID)} wird ein \gls{apLabel} identifiziert.
%		Die \textit{BSSID} entspricht meist der \gls{macLabel}-Adresse des \gls{apLabel}'s und wird nicht manuell vergeben.\footcite{Understanding_the_Network_Terms_SSID_BSSID_and_ESSID_Technical_Documentation_Support_2015-04-10}},
%	first={Basis Service Set ID (BSSID)},
%	long={Basis Service Set ID}
%}


% Example for combined glossary and acronym: http://tex.stackexchange.com/questions/8946/how-to-combine-acronym-and-glossary
%\newglossaryentry{api11}
%{
%	name={API},
%	description={An Application Programming Interface (API) is a particular set
%		of rules and specifications that a software program can follow to access and make use of the services and resources provided by another particular software program that implements that API},
%	first={Application Programming Interface (API)},
%	long={Application Programming Interface}
%}

%% use \gls{api}

% use it with \gls{glos:DVD}
% use plural with \glspl{thinClientLabel}