% !TeX encoding=utf8
% !TeX spellcheck = de_CH_frami

%%% --- Acronym definitions
\IfDefined{newacronym}{%
% examples

% our used acronyms
% zhaw internal
\newacronym{zhawLabel}{ZHAW}{Zürcher Hochschule für Angewandte Wissenschaften}
\newacronym{ebsLabel}{EBS}{Einschreibe und Bewertungssystem der ZHAW}

% it general
%\newacronym{macLabel}{MAC}{Media Access Control}
%\newacronym{asciiLabel}{ASCII}{American Standard Code for Information Interchange}
%\newacronym{httpsLabel}{HTTPS}{Hypertext Transfer Protocol Secure}

\newacronym{ieeeLabel}{IEEE}{Institute of Electrical and Electronics Engineers}
\newacronym{usbLabel}{USB}{Universal Serial Bus}
\newacronym{nfcLabel}{NFC}{Near Field Communication}
\newacronym{pinLabel}{PIN}{Personal Identification Number}
\newacronym{lanLabel}{LAN}{Local Area Network}
\newacronym{wlanLabel}{WLAN}{Wireless Local Area Network}
\newacronym{apLabel}{AP}{Access Point}


%network
%\newacronym{arpLabel}{ARP}{Address Resolution Protocol}
%\newacronym{rtsLabel}{RTS}{Ready to send}
%\newacronym{ctsLabel}{CTS}{Clear to send}

% general
\newacronym{mclLabel}{MC Link}{Multi-Communicator Links}
\newacronym{sigLabel}{SIG}{Bluetooth Special Interest Group}

\newacronym{brLabel}{BR}{Basic Rate}
\newacronym{edrLabel}{EDR}{Enhanced Data Rate}
\newacronym{hsLabel}{HS}{High Speed}
\newacronym{leLabel}{LE}{Low Energy}
\newacronym{sspLabel}{SSP}{Secure Simple Pairing}
\newacronym{oobLabel}{OOB}{Out Of Band}

\newacronym{mitmLabel}{MITM}{Man-In-The-Middle}
\newacronym{scoLabel}{SCO}{Synchronous Connection-Oriented}
\newacronym{aclLabel}{ACL}{Asynchronous Connectionless}
\newacronym{aesccmLabel}{AES-CCM}{Advanced Encryption Standard - Counter with CBC-MAC (cipher block chaining message authentication code)}
\newacronym{crcLabel}{CRC}{Cyclic Redundancy Check}
\newacronym{fecLabel}{FEC}{Forward Error Correction}




%\newacronym{ivLabel}{IV}{initialization vector}


% ble
\newacronym{bleLabel}{BLE}{Bluetooth Low Energy}
\newacronym{hspLabel}{HSP}{Headset Profile}
\newacronym{gavdpLabel}{GAVDP}{Generic Audio/Video Distribution Profile}
\newacronym{hdpLabel}{HDP}{Health Device Profile}
\newacronym{gattLabel}{GATT}{Generic Attribute Profile}







% alternatives
\newacronym{irdaLabel}{IrDA}{Infrared Data Association}
\newacronym{irLabel}{IR}{Infrarot}
\newacronym{losLabel}{LOS}{Line-of-Sight}

\newacronym{rfidLabel}{RFID}{Radio-frequency identification}

% summary
\newacronym{iotLabel}{IoT}{Internet of Things}

}

%%% --- Symbol list entries

%\newglossaryentry{symb:Pi}{%
%  name=$\pi$,%
%  description={mathematical constant},%
%  sort=symbolpi, type=symbolslist%
%}


%%% --- Glossary entries

\newglossaryentry{glos:ioLabel}{
	name={I/O},
	description={Input/Output (I/O) bezeichnet Möglichkeiten für die Kommunikation zwischen Mensch und Maschine. (Bsp.: Bildschirm, Tastatur, Drucker,...)}
}

\newglossaryentry{glos:wearableLabel}{
	name={Wearable},
	plural={Wearables},
	description={Mit \textit{Wearables} (engl. für "`tragbar"') werden technische Geräte bezeichnet, die man im Alltag angenehm bei oder an sich tragen kann (z.B. eine Smartwatch).}
}

\newglossaryentry{glos:roundRobinLabel}{
	name={Round-Robin},
	description={\textit{Round-Robin} (engl. für "`Rundlauf"') bezeichnet eine gleichmässige, meist wiederholende Verteilung von Ressourcen auf mehrere Objekten.\footcite{What_is_round_robin_Definition_from_WhatIs_2015-04-24}}
}

\newglossaryentry{glos:clockLabel}{
	name={Clock},
	description={In der Übertragung definiert der \textit{Clock} die Periodizität in der ein Signal abgetastet werden muss um die Informationen zu empfangen und erfolgreich zu decodieren.}
}

\newglossaryentry{glos:connectionlessCommunicationLabel}{
	name={verbindungsloser Datenaustausch},
	description={Bei einem \textit{verbindungslosen Datenaustausch} werden alle Pakete (Nachrichten) einzeln adressiert und transportiert. Dazu wird keine direkte Verbindung zwischen den Kommunikationspartnern aufgebaut.
	Es können verschiedene Kommunikationspfade für die Übertragung verwendet werden und die Verbindung ist skalierbar. Das populärste Beispiel eines verbindungslosen Netzwerkes ist das TCP/IP-Protokoll, welches im Internet oft Verwendung findet.\\
	Das Gegenteil bildet der \textit{verbindungsorientierte Datenaustausch}, bei der vor der effektiven Datenübertragung eine Verbindung aufgebaut und anschliessend wieder abgebaut werden muss.}
}








%Glossary entries used also as acronyms
%\newglossaryentry{ssidLabel}{
%	name={SSID},
%	description={Mit der \textit{Service Set ID (SSID)} wird ein gesamtes Netzwerk identifiziert. Die \textit{SSID} entspricht dem sichtbaren Netzwerknamen und kann sich über mehrere \gls{apLabel}'s erstrecken.\footcite{Understanding_the_Network_Terms_SSID_BSSID_and_ESSID_Technical_Documentation_Support_2015-04-10}},
%	first={Service Set ID (SSID)},
%	long={Service Set ID}
%}
%
%\newglossaryentry{bssidLabel}{
%	name={BSSID},
%	description={Mit der \textit{Basis Service Set ID (BSSID)} wird ein \gls{apLabel} identifiziert.
%		Die \textit{BSSID} entspricht meist der \gls{macLabel}-Adresse des \gls{apLabel}'s und wird nicht manuell vergeben.\footcite{Understanding_the_Network_Terms_SSID_BSSID_and_ESSID_Technical_Documentation_Support_2015-04-10}},
%	first={Basis Service Set ID (BSSID)},
%	long={Basis Service Set ID}
%}


% Example for combined glossary and acronym: http://tex.stackexchange.com/questions/8946/how-to-combine-acronym-and-glossary
%\newglossaryentry{api11}
%{
%	name={API},
%	description={An Application Programming Interface (API) is a particular set
%		of rules and specifications that a software program can follow to access and make use of the services and resources provided by another particular software program that implements that API},
%	first={Application Programming Interface (API)},
%	long={Application Programming Interface}
%}

%% use \gls{api}

% use it with \gls{glos:DVD}
% use plural with \glspl{thinClientLabel}