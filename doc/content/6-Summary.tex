\chapter{Schlussfolgerung}

%Im Rahmen der Arbeit wurden viele interessante Fakten zum Thema Wireless gesammelt.
%
%Durch den Allgemeinen Teil (\chapref{ch:general}) wurden verschiedene grundlegende Funktionen abgedeckt, welche die Basis für das technische Grundverständnis jedes Wireless-Netzwerkes bilden.
%
%Beim \gls{wepLabel}-Standard (\chapref{ch:wep}) wird einen Einblick in eine älteres und unsicheres Protokoll gewährt.
%Besonders im Angriffsteil (\cref{sec:wepAttack}) kann sich jeder selbst von der Schwachheit von \gls{wepLabel} überzeugen.
%
%Im grössten Teil der Arbeit, der den \gls{wpaLabel}-Standard (\chapref{ch:wpa}) behandelt, wurde nebst dem theoretischen viel Zeit für Angriffen und deren Programmen investiert.
%Da dieses Kapitel den aktuell verwendeten Standard beschreibt, sind darin vorhandene Informationen am Nützlichsten.


\section{Persönliches Fazit}
%Zum Denken angeregt hat mich, dass OS X bereits Sniffing Programme von Apple selbst anbietet.
%
%Nach meiner Recherche kann ich soweit behaupten, dass der Standard-Schlüssel eines Swisscom Routers als sicher gilt, da mit \gls{glos:bruteforceLabel}-Attacken nur mit extremem Glück ein Erfolg verbucht werden kann.
%Auch wurden mir die Sicherheitsmängel der noch oft verwendeten \gls{wpsLabel}-Funktionen bewusst.
%
%Das Thema Wireless hat mich sehr fasziniert, da es ein viel genutzter, uns immer häufig umgebender, aber trotzdem nicht sichtbarer Teil unseres Lebens ausmacht.
%Egal ob im Privaten oder Geschäftlichen Bereich, Wireless ist überall sehr verbreitet.
%Jeder nutzt es, doch so mancher hat keine Ahnung was sich dahinter alles verbirgt.
%
%Ich genoss es, im Rahmen dieser Arbeit, einen grösseren Einblick in dieses Gebiet zu erlangen.
