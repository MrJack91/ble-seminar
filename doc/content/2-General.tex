\chapter{Allgemein Bluetooth}
\label{ch:general}
%Dieses einleitende \chapref{ch:general} gibt einen groben Überblick zum Thema Wireless. Es wird auf Modis, Paket Typen, Paket Adressen, Verschlüsselungstechniken (allgemein) und Angriffe eingegangen.

\section{Geschichte}
Voraussetzung für das heutige weit verbreitete Bluetooth, schuf der Physiker Dr. Johan Ullman.
Bereits 1989 präsentierte er seine erste Erfindung für ein kabelloses Headset.\footcite{Bluetooth_Wikipedia_2015-04-17}

1994 wurden die Erfindungen von der Firma \textit{Ericsson Mobile}\footcite{Ericsson_2015-04-17} aufgenommen.
Mit Hilfe von Intel, Nokia, IBM und Toshiba wurde im Jahr 1998, die neue Technologie durch die \gls{sigLabel}, veröffentlicht.
\footcite{Bluetooth_Special_Interest_Group_Wikipedia_2015-04-17}\footcite{The_history_of_Bluetooth_Ericsson_History_2015-04-17}

Die Bezeichnung der Technologie lautete während der Entwicklung \gls{mclLabel}, wurde aber auf Grund der Faszination eines Mitentwicklers an Harald Bl{\aa}tand, (engl. \textit{Bluetooth}, \textit{Blauzahn}), ehemaliger König von Dänemark während des 10. Jahrhunderts, auf Bluetooth geändert.

Im Jahr 2000 wurde das erste Headset mit Bluetooth verkauft.

\section{Technische Spezifikationen}
\label{sec:general_specs}
Im folgenden \secref{sec:general_specs} stammen mehrere Informationen aus Wikipedia.\footcite{Bluetooth_Wikipedia_2015-04-17}

\subsection{Versionen und Geschwindigkeiten}
Die Version 1.0 wurde 1999 veröffentlicht, sie beinhaltete noch viele Fehler.

Über die Jahren haben sich vier verschiedene Versionen herauskristallisiert:\footcite{Bluetooth_low_energy_Wikipedia_2015-04-17}\footcite{Our_History_Bluetooth_Technology_Website_2015-04-17}
\begin{table}[H]
% style
\small\sffamily\renewcommand{\arraystretch}{1.4}
% caption
\captionabove{Bluetooth Geschwindigkeiten mit Versionen}
\begin{tabular}{llrrl}
\toprule
	Version & Beizeichnung & Geschw. & effektive Geschw.  & Jahr\\
\midrule
	1.2 & \gls{brLabel} & 1\,MBit/s & 0.7\,MBit/s & 2003 \\
	2.0 & \gls{edrLabel} & 3\,MBit/s & 2.1\,MBit/s & 2004 \\
	3.0 & \gls{hsLabel} & 24\,MBit/s & 2.1\,MBit/s & 2009\\
%	4.0 & \gls{hsLabel} & 24\,MBit/s & -  & 2010 \\
	4.0 & \gls{leLabel} Erweiterung & 1\,MBit/s &  0.27\,MBit/s & 2010 \\
\bottomrule
\end{tabular}
\end{table}
% \footcite{How_Bluetooth_Creates_a_Connection_HowStuffWorks_2015-04-17}



\subsection{Physikalische Übertragung}
Die Übertragung des klassischen Bluetooth erfolgt auf den Frequenzen 2.4 bis 2.480\,GHz, welche in 79 Kanäle à 1\, MHz unterteilt werden.
Bei Bluetooth 4.0 halbiert sich die Anzahl Kanäle, da pro Kanal 2\, MHz zugeteilt sind.

Es wird die Funktechnologie des \textit{frequency-hopping spread spectrum} verwendet.
Dabei wird jedes Paket auf einem anderen Kanal versendet.
Der Kanal wird 1\'600 pro Sekunde gewechselt.\footcite{Bluetooth_Wikipedia_2015-04-17}

Durch das \textit{frequency-hopping} ergeben sich drei wesentliche Vorteile:\footcite{Frequency-hopping_spread_spectrum_-_Wikipedia_2015-04-17}
\begin{itemize}
	\item \textbf{Anzahl Verbindungen:} Die Übermittlung eines Pakets blockiert einen Kanal nur für kurze Zeit.
		Anschliessend kann der Kanal bereits wieder für die nächste Übertragung verwendet werden.
		So steigt die Effizienz über die ganze Bandbreite.
	\item \textbf{Abhörsicherheit:} Durch das Wechseln der Kanäle, wird das Abhören massiv erschwert, da dem Angreifer die Folge der verwendeten Kanäle nicht bekannt ist.
	\item \textbf{Rauschen:} Durch die Wechsel wird das Rauschen unterdrückt.
\end{itemize}

\begin{framed}
	\textbf{Information:} Das \textit{frequency-hopping spread spectrum} wurde von Hedy Lamarr, einer österreich-amerikanischen Frau, während des zweiten Weltkrieges erfunden. Zudem war sie eine erfolgreiche Schauspielerin.\footcite{Hedy_Lamarr_Wikipedia_2015-04-27}
\end{framed}


\subsection{Klassen (Reichweite)}
Bluetooth Geräte können abhängig ihrer Reichweite in folgende drei Klassen eingeteilt werden:
\begin{itemize}
	\item \textbf{Klasse 1:} Distanz bis zu 100\,m
	\item \textbf{Klasse 2:} Distanz bis zu 10\,m
	\item \textbf{Klasse 3:} Distanz bis zu 1\,m
\end{itemize}
Die meisten gewöhnlichen Geräte (wie Mobiltelefone, Computers, Headsets, etc.) entsprechen der \textit{Klasse 2}.



\section{Kommunikation}
Bluetooth ist ein paketbasiertes Protokoll, mit einem \textit{Master-Slave} Prinzip.

Pro Netzwerk (\textit{Piconet}) kann ein Master mit bis zu sieben aktiven Slaves kommunizieren.
Der Master definiert welcher Slave wann antworten darf (meist via \gls{glos:roundRobinLabel}).
Folglich muss ein Slave immer aktiv zuhören, um seinen zugeteilten Sende-Slot nutzen zu können. Die Funktion des Masters ist weniger aufwändig, als die eines Slaves.
Nebst den aktiven Slaves, können bis zu 255 Slaves in den \textit{PARK}-Modus (siehe \cref{subsec:energymode}) gesetzt werden, aus dem sie jederzeit reaktiviert werden können.
\footcite{Piconet_-_Wikipedia_2015-04-18}


Die Rollen (Master und Slave) können getauscht werden.
Dies geschieht zum Beispiel, wenn via Headset eine Verbindung zu einem Mobiltelefon aufgebaut wird:
zuerst übernimmt das Headset die Rolle des Masters, übergibt die aber sobald wie möglich dem Telefon.

Der \textit{Master} definiert für all seine Slaves den \textit{Clock} ($312.5\,\mu s$).
Jeweils zwei Clocks ergeben einen \textit{Slot}.
Der Master beginnt eine Sendung immer auf einen geraden Slot.
Für die Slaves stehen die ungeraden Slots als Start einer Sendung zur Verfügung.

Ein Paket kann 1, 3 oder 5 Slots lang sein.

Bluetooth definiert mehrere Piconets zu einem \textit{Scatternet}, in dem Geräte nebst dem Master auch die Rolle des Slaves in anderen Netzen übernehmen können.

\subsection{Asynchron und synchrone Kommunikation}
Die Kommunikation kann entweder \gls{scoLabel} oder \gls{aclLabel} erfolgen.
\gls{scoLabel} wird für den Austausch von Sprachdateien (64\,kbit/s) verwendet, wobei bis zu drei duplex Verbindungen zur gleichen Zeit aktiv sein können.
\gls{aclLabel} setzt ein speicherndes Verhalten der Endgeräte voraus (analog zum Internet). Die asynchrone Verbindung erlaubt eine \textit{symmetrische} (mit 2 x 433,9\,kbit/s) und eine \textit{asymmetrische} (mit 706,25 und 57,6\,kbit/s) verteilte Übertragungsgeschwindigkeit.
\footcite{Nahfunktechnik_in_Smartphones_FAQ_cio.de_2015-04-24}


\subsection{Energiesparmodi}
\label{subsec:energymode}
Da klassisches Bluetooth bereits seit Beginn eine stromsparende Nutzung anvisiert, gibt es folgende drei Energiesparmodi:
\begin{itemize}
	\item \textbf{HOLD-Modus:} Hier kann vorläufig eine Abwesenheit mitgeteilt werden, während dem keine Daten Empfangen werden. So können während dieser Zeit bewusst andere Aufgaben erledigt werden.
	\item \textbf{SNIFF-Modus:} Im SNIFF-Modus werden periodische Aktivitäten reduziert. Dieser Modus wird sehr oft eingesetzt um den Energieverbrauch zu senken.
	\item \textbf{PARK-Modus:} Das Gerät bleibt zwar synchronisiert, kann jedoch während diesem Modus nicht am Datenverkehr teilnehmen. Dieser Modus wird in der Praxis kaum verwendet.
\end{itemize}


\section{Verbindungsaufbau}
Sobald sich ein Gerät im \textit{Discoverable Mode} befindet, sendet es folgende Daten auf Anfrage: \textit{Geräte Name}, \textit{Geräte Klasse}, \textit{Liste der unterstützten Services} und \textit{weitere technische Informationen}.

Jedes Gerät kann die Informationen jedes anderen Gerätes im \textit{Discoverable Mode} jederzeit verlangen und direkt ein Verbindung aufbauen.

Technisch wird jedes Gerät mit einer 48-Bit Adresse identifiziert. Der Benutzer sieht jedoch lediglich die nicht eindeutigen und selbst definierbare Bezeichnung. Viele Hersteller setzen diese Bezeichnung standardmässig auf die Modellbezeichnung, was dazu führt, dass oft mehrmals gleiche Bezeichnungen angezeigt werden.

Um nun eine Verbindung herzustellen, wird das gewünschte Gerät in den \textit{Scan-Modus} gesetzt.
Dabei wird auf 32 Hop-Frequenzen, während 2.56\,s nach möglichen Slaves gesucht. Dazu werden sogenannte Inquiry-Messages (Erkundigungsnachrichten) verschickt.
Wenn Slaves gefunden werden, wird ihnen per Page-Messages auf 16 Kanälen eine Verbindung angeboten.

Falls keine Verbindung zustande kommt, wird der Suchprozess auf weiteren 16 Hop-Frequenzen fortgefahren.

Der durchschnittliche Verbindungsaufbau dauert 1.28\,s.\footcite{Bluetooth_de_Wikipedia_2015-04-18}

Ab Bluetooth 3.0 ist zusätzlich ein verbindungsloser Datenaustausch möglich.

\section{Profile}
\todo{add something, you have to correct it}
Bluetooth definiert verschiedene \textit{Bluetooth Profile} um Daten zu übertragen.
Für ein erfolgreicher Datenaustausch müssen jeweils alle Geräte das zu verwendende Profil unterstützen.

Selbstsprechende Beispiel für häufig verwendete Profile sind: \gls{hspLabel}, \gls{gavdpLabel}, \gls{hdpLabel}.\footcite{List_of_Bluetooth_profiles_Wikipedia_2015-04-27}

Eine vollständige Sammlung der standardisierten Profilen ist auf der Website zu Bluetooth verfügbar.\footcite{Profiles_Overview_Bluetooth_Development_Portal_2015-04-27}


\section{Sicherheit}
\todo{add intro security}

\subsection{Pairing, Bonding}
Viele Services benötigen nebst einem Verbindungsaufbau, ein erfolgreiches \textit{Bonding}, dass durch ein \textit{Pairing} (Koppelung) zustand kommt.
Dabei wird initial die Identität der beiden Geräte verifiziert, so dass künftig Services genutzt werden können, ohne dass eine Benutzer Interaktion erforderlich ist.

Seit der Bluetooth Version 2.1 wird das \gls{sspLabel} standardmässig verwendet. Es unterstützt folgende vier Modi:\footcite{Bluetooth_Wikipedia_2015-04-17}
\begin{itemize}
	\item \textbf{Just works:} Es wird keine Interaktion des Benutzers benötigt. Dies wird oft für Geräte ohne \gls{glos:ioLabel} Möglichkeiten eingesetzt, und bietet keinen Schutz gegen \gls{mitmLabel}-Angriffe.
	\item \textbf{Numeric Comparison:} Auf den Displays beiden Geräten wird ein Zahl angezeigt, die übereinstimmen muss. So wird sichergestellt, dass kein \gls{mitmLabel}-Angriff vorliegt.
	\item \textbf{Passkey Entry:} Hier muss mind. ein \gls{pinLabel} des einen Gerätes auf dem anderen eingegeben werden. Somit wird \gls{mitmLabel}-Angriff verunmöglicht.
	\item \textbf{\gls{oobLabel}:} Diese Methode wird meistens über \gls{nfcLabel} realisiert.
		\gls{oobLabel} schützt nicht per Definition vor \gls{mitmLabel}-Angriffen, aber erschwert sie erheblich. Der Angreifer muss ein weiteres Medium abhören können, was z.B. bei \gls{nfcLabel} nur in extrem kleinem Radius möglich ist.
\end{itemize}


\subsection{Sicherheitsmodi}
Im Bluetooth-Standard werden vier Sicherheitsmodi definiert:\footcite{Security_Bluetooth_Development_Portal_2015-04-24}
\begin{itemize}
	\item \textbf{Mode 1: Non-Secure} In diesem Modus werden keine Sicherheitsvorsichtsmassnahmen unterstützt.
	\item \textbf{Mode 2: Service-Secure} In diesem Modus erfolgt die Sicherheit auf dem \textit{Application Layer} des Services.
	\item \textbf{Mode 3: Link-Secure} Dieser Modus unterstützt eine Authentifizierung auf dem \textit{Link Layer}.
	\item \textbf{Mode 4: Link-Secure encrypted} Dieser Modus unterstützt ebenfalls die Authentifizierung auf dem \textit{Link Layer}, wobei hier der Schlüssel-Austausch explizit verschlüsselt erfolgen muss.
\end{itemize}

\subsection{Verschlüsselung}
\todo{add something, you have to correct it}
Bei klassischem Bluetooth wird die Stromchiffre \textit{E0}\footcite{E0_cipher_Wikipedia_2015-04-27} verwendet.
Dazu werden folgende fünf Werte benötigt:
\footcite{LE_Security_Bluetooth_Development_Portal_2015-04-25}
\footcite{Bluetooth_Communication_Hybrid_Encryption_2015-04-25}
\footcite{BluetoothSecurity_Washington_2015-04-25}

\begin{table}[H]
	% style
	\small\sffamily\renewcommand{\arraystretch}{1.4}
	% caption
	\captionabove{Verschlüsselungs-Werte}
	\begin{tabular}{p{0.25\linewidth}lp{0.5\linewidth}}
		\toprule
		Parameter & Länge & Beschreibung \\
		\midrule
		BD\_ADDR & 48 Bit & Entspricht der eindeutigen Geräte-Adresse (definiert von \gls{ieeeLabel}). \\
		\gls{pinLabel} & bis zu 128 Bit & \gls{pinLabel} der auf beiden Geräten übereinstimmen muss. Entweder hat ein Gerät einen fix hinterlegten \gls{pinLabel} oder er kann selbst definiert werden.\\
		Link Key & 128 Bit  & Wird jeweils von zwei paired Geräte verwendet. Kann auch \textit{Private Authentication Key} genannt werden.\\
		Private Encryption Key & 8 -- 128 Bit & Pro Übertragung (Session) wird ein Encryption Key aus dem Private Authentication Key abgeleitet. Die Länge wird durch die zwei Geräte festgelegt, wobei beide ein Minimum sowie ein Maximum definieren können. Falls sich keine Übereinstimmung ergibt, muss abhängig des Services entschieden werden, ob trotzdem eine Verbindung zu Stand kommen darf. \\
		RAND & 128 Bit & Der RAND-Wert wechselt häufig und wird jeweils für die Key Generierung verwendet.\\
		\bottomrule
	\end{tabular}
\end{table}

Mit Hilfe des \gls{pinLabel}'s, des RAND's und der BD\_ADDR, wird ein \textit{Initialization Key} berechnet.
Für die Authentifizierung müssen beide Geräte eine zufällige Zahl des anderen korrekt berechnen können.
Bei Erfolg, wird die BD\_ADDR vom \textit{Service Manager} des Gerätes als authentifiziert abgelegt. Bei Misserfolg, entstehen exponentielle Wartezeiten.

Anschliessend wird ein gültiger Link Key für die effektive Verschlüsselung der Daten ausgemacht.

Bei \gls{bleLabel} erfolgt eine Verschlüsselung via \gls{aesccmLabel}.

\subsection{Error Correction}
Die Daten werden entweder mit \gls{fecLabel}\footcite{Forward_error_correction_Wikipedia_2015-04-27} (beim klassischen Bluetooth) oder mit \gls{crcLabel}\footcite{Cyclic_redundancy_check_Wikipedia_2015-04-27} (bei \gls{bleLabel}) auf Übermittlungsfehler überprüft.

