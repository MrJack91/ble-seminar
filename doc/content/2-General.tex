\chapter{Allgemein Bluetooth}
\label{ch:general}
%Dieses einleitende \chapref{ch:general} gibt einen groben Überblick zum Thema Wireless. Es wird auf Modis, Paket Typen, Paket Adressen, Verschlüsselungstechniken (allgemein) und Angriffe eingegangen.

\section{Geschichte}
Voraussetzung für das heutige weit verbreitete Bluetooth, schuf der Physiker Dr. Johan Ullman.
Bereits 1989 präsentierte er seine erste Erfindung für ein kabelloses Headset.\footcite{Bluetooth_Wikipedia_2015-04-17}

1994 wurden die Erfindungen von der Firma \textit{Ericsson Mobile}\footcite{Ericsson_2015-04-17} aufgenommen.
Mit Hilfe von Intel, Nokia, IBM und Toshiba wurde im Jahr 1998, die neue Technologie durch die \gls{sigLabel}, veröffentlicht.
\footcite{Bluetooth_Special_Interest_Group_Wikipedia_2015-04-17}\footcite{The_history_of_Bluetooth_Ericsson_History_2015-04-17}

Die Bezeichnung der Technologie lautete während der Entwicklung \gls{mclLabel}, wurde aber auf Grund der Faszination eines Mitentwickler an Harald Bl{\aa}tand, (engl. \textit{Bluetooth}, \textit{Blauzahn}) König von Dänemark während des 10. Jahrhunderts.

Im Jahr 2000 wurde das erste Headset mit Bluetooth geliefert.

\section{Technische Spezifikationen}
\label{sec:general_specs}
Im folgenden \secref{sec:general_specs} stammen mehrere Informationen aus Wikipedia.\footcite{Bluetooth_Wikipedia_2015-04-17}

\subsection{Versionen und Geschwindigkeiten}
Die Version 1.0 wurde 1999 veröffentlicht, sie beinhaltete noch viele Fehler.

Über die Jahren haben sich vier verschiedene Versionen herauskristallisiert:\footcite{Bluetooth_low_energy_Wikipedia_2015-04-17}\footcite{Our_History_Bluetooth_Technology_Website_2015-04-17}
\begin{table}[H]
% style
\small\sffamily\renewcommand{\arraystretch}{1.4}
% caption
\captionabove{Bluetooth Geschwindigkeiten mit Versionen}
\begin{tabular}{llrrl}
\toprule
	Version & Beizeichnung & Geschw. & effektive Geschw.  & Jahr\\
\midrule
	1.2 & \gls{brLabel} & 1\,MBit/s & 0.7\,MBit/s & 2003 \\
	2.0 & \gls{edrLabel} & 3\,MBit/s & 2.1\,MBit/s & 2004 \\
	3.0 & \gls{hsLabel} & 24\,MBit/s & 2.1\,MBit/s & 2009\\
	4.0 & \gls{hsLabel} & 24\,MBit/s & -  & 2010 \\
	4.0 & \gls{leLabel} & 100\,kb/s &  - & 2010 \\
\bottomrule
\end{tabular}
\end{table}
% \footcite{How_Bluetooth_Creates_a_Connection_HowStuffWorks_2015-04-17}



\subsection{Physikalische Übertragung}
Die Übertragung des klassischen Bluetooth erfolgt auf den Frequenzen 2.4 bis 2.480\,GHz, welche in 79 Kanäle à 1\, MHz unterteilt werden.
Bei Bluetooth 4.0 halbiert sich die Anzahl Kanäle, da pro Kanal 2\, MHz zugeteilt sind.

Es wird die Funktechnologie des \textit{frequency-hopping spread spectrum} verwendet.
Dabei wird jedes Paket auf einem anderen Kanal versendet.
Der Kanal wird 1600 pro Sekunde gewechselt.\footcite{Bluetooth_Wikipedia_2015-04-17}

Durch das \textit{frequency-hopping} ergeben sich drei wesentliche Vorteile:\footcite{Frequency-hopping_spread_spectrum_-_Wikipedia_2015-04-17}
\begin{itemize}
	\item \textbf{Anzahl Verbindungen:} Die Übermittlung eines Pakets blockiert einen Kanal nur für kurze Zeit. Anschliessend kann der Kanal bereits für wieder für die nächste Übertragung verwendet werden. So steigt die Effizienz über der ganzen Bandbreite.
	\item \textbf{Abhörsicherheit:} Da die Kanäle immer wechseln, wird das Abhören massiv erschwert, da dem Angreifer die Folge der verwendeten Kanäle nicht bekannt ist.
	\item \textbf{Rauschen:} Durch die Wechsel wird das Rauschen unterdrückt.
\end{itemize}


\subsection{Klassen (Reichweite)}
Bluetooth Geräte können abhängig ihrer Reichweite in folgende drei Klassen eingeteilt werden:
\begin{itemize}
	\item \textbf{Klasse 1:} Distanz bis zu 100\,m
	\item \textbf{Klasse 2:} Distanz bis zu 10\,m
	\item \textbf{Klasse 3:} Distanz bis zu 1\,m
\end{itemize}
Die meisten gewöhnlichen Geräte (wie Mobiltelefone, Computers, Headsets, etc.) fallen unter die Klasse 2.



\section{Kommunikation}
Bluetooth ist ein Paket basiertes Protokoll, mit einem \textit{Master-Slave} Prinzip.

Pro Netzwerk (\textit{Piconet}) kann ein Master mit bis zu sieben aktiven Slaves kommunizieren.
Der Master definiert welcher Slave wann antworten darf (meist via Round-Robin).\todo{Round-Robin Glossary} 
Nebst den aktiven können bis zu 255 Slaves in den \textit{PARK}-Modus (siehe \cref{subsec:energymode}) gesetzt werden, aus dem sie jederzeit wieder aktiviert werden können.
\footcite{Piconet_-_Wikipedia_2015-04-18}
Folglich muss ein Slave immer aktiv zuhören, um seinen zugeteilten Sende-Slot nutzen zu können. Die Funktion des Masters ist weniger aufwändig, als die eines Slaves.

Die Rollen von Master und Slave können getauscht werden.
Dies geschieht zum Beispiel, wenn via Headset eine Verbindung zu einem Mobiltelefon aufgebaut wird.
Zuerst übernimmt das Headset die Rolle des Master, übergibt die aber sobald als möglich dem Telefon.

Der \textit{Master} definiert für all seine Slaves den \textit{Clock} ($312.5\,\mu s$).
Jeweils zwei Clocks ergeben einen \textit{Slot}.
Der Master beginnt eine Sendung immer auf einen geraden Slot.
Für die Slaves stehen die ungeraden Slots als Start einer Sendung zur Verfügung.

Ein Paket kann 1, 3 oder 5 Slots lang sein.

Bluetooth definiert mehrere Piconets zu einem Scatternet, in welchem Geräte nebst dem Master noch die Rolle des Slaves in anderen Netzen übernehmen können.

\subsection{Asynchron und synchrone Kommunikation}
Die Kommunikation kann \gls{scoLabel} oder \gls{aclLabel} erfolgen.
\gls{scoLabel} wird für den Austausch von Sprachdateien (64\,kbit/s) verwendet.
\gls{aclLabel} setzt ein speicherndes Verhalten der Endgeräte voraus (analog zum Internet).

Die synchrone Verbindung erlaubt eine \textit{symmetrische} und eine \textit{asymmetrische} verteilte Übertragungsgeschwindigkeit.


\subsection{Energiesparmodi}
\label{subsec:energymode}
Da klassisches Bluetooth bereits seit Beginn eine stromsparende Nutzung anvisierte, gibt folgende drei Energiesparmodi:
\begin{itemize}
	\item \textbf{HOLD-Modus:} Hier kann vorläufig eine Abwesenheit mitgeteilt werden, während dem keine Daten Empfangen werden. So können während dieser Zeit bewusst andere Aufgaben erledigt werden.
	\item \textbf{SNIFF-Modus:} Im SNIFF-Modus werden periodische Aktivitäten reduziert. Dieser Modus wird sehr oft gebraucht um den Energieverbrauch zu senken.
	\item \textbf{PARK-Modus:} Das Gerät bleibt zwar im synchronisiert, kann jedoch während diesem Modus nicht am Datenverkehr teilnehmen. Dieser Modus wird in der Praxis kaum gebraucht.
\end{itemize}


\section{Verbindungsaufbau}
Sobald sich ein Gerät im \textit{Discoverable Mode} befindet, sendet es auf Anfrage \textit{Device Name}, \textit{Device Class}, \textit{List of Services}, \textit{Technical Information}.

Jedes Gerät kann die Informationen jedes anderen Gerätes im \textit{Discoverable Mode} jederzeit verlangen und eine direkte Verbindung aufbauen.

Technisch wird jedes Device mit einer 48-Bit Adresse identifiziert. Der Benutzer sieht jedoch lediglich die nicht eindeutigen und selbst definierte Bezeichnung. Viele Hersteller setzen die Bezeichnung standardmässig auf die Modellbezeichnung, dass dazu führt, dass oft mehrmals die gleiche Bezeichnung angezeigt wird.

Um nun eine Verbindung herzustellen wird das gewünschte Gerät in den \textit{Scan-Modus} versetzt. Dabei wird auf 32 Hop-Frequenzen, während 2.56\,s nach möglichen Slaves gesucht. Dazu werden sogenannte Inquiry-Message (Erkundigungsnachrichten) verschickt. Werden Slaves gefunden, wird ihnen per Page-Message  auf 16 Kanälen eine Verbindung angeboten. Kommt eine Verbindung zustande, sind die Geräte erfolgreich verbunden.

Falls keine Verbindung zustande kommt, wird auf weiteren 16 Hop-Frequenzen weiter gesucht.

Der Durchschnittliche Verbindungsaufbau dauert 1.28\,s.\footcite{Bluetooth_de_Wikipedia_2015-04-18}

Ab Bluetooth 3.0 ist auch ein verbindungsloser Datenaustausch möglich.


\section{Sicherheit}

\subsection{Sicherheitslevel}
% device, service, no
% http://de.wikipedia.org/wiki/Bluetooth
\footcite{Bluetooth_de_Wikipedia_2015-04-18}

\subsection{Pairing, Bonding}
Viele Service benötigen nebst einem Verbindungsaufbau, ein erfolgreiches \textit{Bonding}, dass durch ein \textit{Pairing} (Koppelung) zustand kommt.
Dabei wird initial die Identität der beiden Geräte überprüft, so dass in Zukunft Services genutzt werden können, ohne das User Interventionen benötigt werden.

Seit der Bluetooth Version 2.1 ist das \gls{sspLabel} Standard. Es unterstützt folgende vier verschiedene Modi:\footcite{Bluetooth_Wikipedia_2015-04-17}
\begin{itemize}
	\item \textbf{Just works:} Es wird keine Interaktion des Users benötigt. Dies wird oft für Geräte ohne \gls{glos:ioLabel} Möglichkeiten verwendet, und bietet keinen Schutz gegen \gls{mitmLabel}.
	\item \textbf{Numeric Comparison:} Auf den Displays beiden Geräten wird ein Zahl angezeigt, die übereinstimmen muss. So wird sichergestellt, dass sich \gls{mitmLabel}-Angriff vorliegt.
	\item \textbf{Passkey Entry:} Hier muss mind. ein \gls{pinLabel} des einen Gerätes auf dem anderen eingegeben werden. Ebenso wird hier \gls{mitmLabel}-Angriff verunmöglicht.
	\item \textbf{\gls{oobLabel}:} Diese Methode findet vor allem Verwendung über \gls{nfcLabel}. \gls{oobLabel} schützt nicht per Definition vor \gls{mitmLabel}-Angriffen, aber erschwert sie erheblich. Der Angreifer muss eub weiteres Medium abhören (dies ist z.B. bei \gls{nfcLabel} nur in extrem kleinem Radius möglich).
\end{itemize}


\subsection{Sicherheitsmodi}
Im Bluetooth-Standard werden 3 Sicherheitsmodi definiert:
\begin{itemize}
	\item \textbf{Non-Secure Mode:} In diesem Modus werden keine Sicherheitsvorsichtsmassnahmen unterstützt.
	\item \textbf{Service-Secure Mode:} In diesem Modus erfolgt die Sicherheit auf dem \textit{Application Layer} des Services.
	\item \textbf{Link-Secure Mode:} Dieser Modus unterstützt die Authentifizierung auf dem \textit{Link Layer} und gilt als sichersten Modus.
\end{itemize}

\subsection{Verschlüsselung}
\todo{add someting}


\section{Unterstützte Protokolle}
\todo{add someting}

\subsection{Error Correction}
\todo{add someting}


%http://en.wikipedia.org/wiki/Bluetooth#Setting_up_connections
%http://de.wikipedia.org/wiki/Bluetooth#Verbindungsaufbau
%http://en.wikipedia.org/wiki/Bluetooth_low_energy
%http://electronics.howstuffworks.com/bluetooth1.htm
%http://www.explainthatstuff.com/howbluetoothworks.html
%http://www.medicalelectronicsdesign.com/article/bluetooth-low-energy-vs-classic-bluetooth-choose-best-wireless-technology-your-application
%http://www.link-labs.com/bluetooth-vs-bluetooth-low-energy/
%http://www.quora.com/Is-there-a-difference-between-Bluetooth-4-0-and-Bluetooth-Low-Energy-If-so-what


