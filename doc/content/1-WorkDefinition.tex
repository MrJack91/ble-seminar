\chapter{Beschreibung der Aufgabe}

\section{Einführung}
\subsection{Zielpublikum}
Obwohl im Glossar einige Fachbegriffe erklärt werden, wird für das Verstehen dieser Arbeit technisches Verständnis vorausgesetzt.

Eine konsequente Übersetzung englischer Begriffe würde keinen Beitrag bezüglich Verständlichkeit leisten und wurde daher weggelassen.

\subsection{Ressourcen}
Weitere Dateien, wie der \LaTeX-Source dieses Dokuments, sind auf GitHub\footcite{GitHub_ble-seminar_2015-04-17} verfügbar.

\section{Aufgabenstellung}
% gemäss EBS
Folgende Aufgabenstellung wurde direkt aus dem \gls{ebsLabel} entnommen.

\subsection{Thema}
Die Technologien \textit{klassisches} und \textit{low energy} Bluetooth sollen in der vorliegenden Arbeit beschrieben und verglichen werden.

\subsection{Ausgangslage}
Das klassische Bluetooth gibt es schon seit langer Zeit.
Seit circa 4 Jahren gibt es jedoch auch \gls{bleLabel} für die Masse. Studien erwarten bis 2018 eine 90\,\% Support von \gls{bleLabel} bei Bluetooth-fähigen Mobiltelefonen.
Besonders bei den aufkommenden \glspl{glos:wearableLabel} ist \gls{bleLabel} nicht mehr wegzudenken.

\subsection{Ziel der Arbeit}
Die Arbeit soll folgende Fragen beantworten:
\begin{itemize}
	\item Wie funktioniert klassisches Bluetooth?
	\item Aus welchem Hintergrund wurde \textit{klassisches} Bluetooth entwickelt?
	\item Was sind die Hauptunterschiede zwischen der \textit{klassischen} und der \textit{low energy} Variante (bezüglich Funktionsweise, Stromverbrauch, Kosten und Anwendungsfällen)?
	\item Welche Vorteile bringt \gls{bleLabel}?
	\item Für welche Anwendungsfälle eignet sich \gls{bleLabel} besonders? Für welche nicht?
\end{itemize}

\subsection{Aufgabenstellung}
Im Rahmen der Arbeit sollen folgende Aufgaben recherchiert und durchgeführt werden:
\begin{itemize}
	\item Für \textit{klassisches} Bluetooth, wie auch für \gls{bleLabel}, sollen folgende Themen abgedeckt werden:
	\begin{itemize}
		\item Geschichte
		\item Technische Funktionsweise (Protokoll, Stromverbrauch)
		\item Kosten
		\item Anwendungsfälle
	\end{itemize}
	\item Mögliche Alternativen zu \gls{bleLabel} werden erwähnt (mit Vor- und Nachteilen)
\end{itemize}

\subsection{Erwartete Resultate}
Das aus der Arbeit resultierende Dokument soll einen Überblick über \textit{klassisches} Bluetooth und \gls{bleLabel} ermöglichen.

\subsection{Eingrenzung}
Es wird auf Bluetooth im Allgemeinen (klassisches Bluetooth) (\chapref{ch:general}) und auf \gls{bleLabel} (\chapref{ch:ble}) und deren Funktionsweise, sowie physikalischen Eigenschaften eingegangen.

Anschliessend erfolgen verschiedene Alternativen (\chapref{ch:alt}) die ebenfalls eine drahtlose Verbindung anbieten.

\subsection{Abgrenzung}
Die aufgeführten Kapitel sind nicht vollständig, sondern dienen dazu einen fundierten Einblick ins Thema Bluetooth und dessen Funktionsweise zu erlangen.

Besonders im \chapref{ch:alt} werden die Alternativen nur oberflächlich behandelt und dienen zur Übersicht.
