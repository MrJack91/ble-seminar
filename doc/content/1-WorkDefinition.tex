\chapter{Beschreibung der Aufgabe}

\section{Einführung}
\subsection{Zielpublikum}
Trotzdem im Glossar einige Fachbegriffe erklärt werden, wird für das Verstehen dieser Arbeit technisches Verständnis vorausgesetzt.

Eine konsequente Übersetzung englischer Begriffe würde keinen Beitrag bezüglich Verständlichkeit leisten und wurde daher weggelassen.

\subsection{Ressourcen}
Weitere Dateien, wie der \LaTeX-Source dieses Dokuments, ist auf GitHub\footcite{GitHub_ble-seminar_2015-04-17} verfügbar.

\section{Aufgabenstellung}
% gemäss EBS
Folgende Aufgabenstellung wurde direkt aus dem \gls{ebsLabel} entnommen.

\subsection{Thema}
Die Technologien \textit{klassisches} und \textit{low energy} Bluetooth sollen beschrieben und verglichen werden.

\subsection{Ausgangslage}
Schon seit langer Zeit gibt es das klassische Bluetooth. Doch seit ca. 4 Jahren gibt es auch \gls{bleLabel} für die Masse. Studien prophezeien bis 2018 eine 90\,\% Support von \gls{bleLabel} bei Bluetooth-fähigen Mobiltelefonen.
Besonders bei den aufkommenden \glspl{glos:wearableLabel} ist \gls{bleLabel} nicht mehr wegzudenken.

\subsection{Ziel der Arbeit}
Die Arbeit soll folgende Fragen beantworten:
\begin{itemize}
	\item Wie funktioniert klassisches Bluetooth?
	\item Aus welchem Hintergrund wurde \textit{klassisches} Bluetooth entwickelt?
	\item Was sind die Hauptunterschiede zwischen der \textit{klassischen} und der \textit{low energy} Variante (bezüglich Funktionsweise, Stromverbrauch, Kosten und Anwendungsfällen)?
	\item Welche Vorteile bringt \gls{bleLabel}?
	\item Für welche Anwendungsfälle eignet sich \gls{bleLabel} besonders? Für welche nicht?
\end{itemize}

\subsection{Aufgabenstellung}
Im Rahmen der Arbeit sollen folgende Aufgaben recherchiert und durchgeführt werden:
\begin{itemize}
	\item Für \textit{klassisches} Bluetooth, wie auch für \gls{bleLabel}, sollen folgende Themen abgedeckt werden:
	\begin{itemize}
		\item Geschichte
		\item Technische Funktionsweise (Protokoll, Stromverbrauch)
		\item Kosten
		\item Anwendungsfälle
	\end{itemize}
	\item Mögliche Alternativen zu \gls{bleLabel} sollen erwähnt werden (mit Vor- und Nachteilen)
\end{itemize}

\subsection{Erwartete Resultate}
Das aus der Arbeit resultierende Dokument soll einen Überblick über \textit{klassisches} Bluetooth und \gls{bleLabel} ermöglichen.

\subsection{Eingrenzung}
\todo{Eingrenzung}
%Es wird auf Wireless im Allgemeinen eingegangen (\chapref{ch:general}).
%Zudem wird Fokus auf den \gls{wepLabel}-Standard (\chapref{ch:wep}), so wie der \gls{wpaLabel}-Standard (\chapref{ch:wpa}) gelegt. Beide Standards werden anhand eines konkreten Beispieles angegriffen.

\subsection{Abgrenzung}
\todo{Abgrenzung}
%Die aufgeführten Kapitel sind nicht vollständig, sondern dienen dazu einen fundierten Einblick ins Thema "`Wireless Security Standard"' und möglicher Angriffsszenarien zu gewinnen.

%Besonders auf den \textit{\gls{wpaLabel} enterprise} Standard wird auf Grund von mangelnder Zeit bewusst nicht eingegangen.
