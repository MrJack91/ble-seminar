\chapter{Alternativen}
\label{ch:alt}
Folgendes Kapitel soll die Vor- und Nachteile von möglichen Alternativen zur Datenkommunikation aufzeigen.

\section{Kabel}
Die simpelste und älteste Methode zur Kommunikation ist das Kabel.

\textbf{Vorteile:} Alleine durch das Medium ist ein Kabel sicherer als z.B. Funk. Zudem kann die Verbindung theoretisch verschlüsselt erfolgen, so dass bei langen, unübersichtbaren Verbindungen das Abhören massiv erschwert wird. Kabel sind meist sehr günstig, schnell, stromsparend und können Standardstecker verwenden (z.B. \gls{usbLabel}).

\textbf{Nachteile:} Die hauptsächlichen Nachteile eines Kabels liegen alle im Bereich Mobilität.
Zu verwendende Kabel müssen mitgetragen und eingesteckt werden.
Oft erzwingen verschiedene Schnittstellen zusätzliche Hardware (weitere Kabel und Adapters), so dass alle Anforderungen abgedeckt werden können.
Für Präsentationen ausserorts, müssen oft mehrere Adapter zur Verfügung stehen, so dass sichergestellt werden kann, dass alles reibungslos funktioniert.
Diese Vielfalt von Hardware wirkt sich natürlich auch finanziell aus. Zu alldem hängen jeweilige Geräte direkt am Kommunikationspartner, was beispielsweise bei Nutzung eines Smartphone kaum vorstellbar wäre.

Die genannten Nachteile stellten ursprünglich die Motivation für drahtlose Kommunikation dar.


\section{Drahtlose Verbindungen}
http://dminc.com/blog/bluetooth-beacons-vs-wifi-vs-nfc/
https://www.themobilestore.in/blog/bluetooth-vs-nfc-vs-wi-fi-direct/
http://www.shoppertrak.mx/blog/wifi-bluetooth-ble-nfc-for-retail-marketing/
http://www.digikey.com/en/articles/techzone/2011/aug/comparing-low-power-wireless-technologies
https://www.phonegurureviews.com/bluetooth-nfc-wifi-direct/
http://mobileworldcapital.com/239/



\subsection{Infrarot}



\subsection{Wireless LAN / Wi-Fi Direct - IEEE 802.11}
\textit{\gls{wlanLabel}} (\gls{ieeeLabel} 802.11) und \textit{Wi-Fi Direct} unterscheiden sich darin, indem bei einem gewöhnlichem \gls{wlanLabel} die ganze Kommunikation von einem \gls{apLabel} gesteuert wird, über den gewöhnlich zudem auch der ganze Datenfluss fliesst. Dies wirkt sich selbst unter optimalen Umständen negativ auf die Datenübertragungsgeschwindigkeit aus.

Wi-Fi Direct erlaubt eine direkte Kommunikation ohne \gls{apLabel}.

\textbf{Vorteile:}
Grundsätzlich gilt das \gls{wlanLabel} als weit verbreitendes und als sehr schnelle Möglichkeit um Daten auszutauschen. Die Reichweite ist mit ca. 100\,m relativ weit, was eine angenehme und freie Nutzung ermöglicht.
\gls{wlanLabel} ist sehr verbreitet und relativ günstig. (Gerade die Verbreitung gilt jedoch nicht für Wi-Fi Direct.)

\textbf{Nachteile:}
Viel Geschwindigkeit über Funk heisst auch hohe Stromverbrauch. Zudem ist Wi-Fi Direct noch nicht so lange bekannt, 



\todo{wifi direct?! Standard, Maximale Geschwindigkeit}


\subsection{NFC}
\gls{nfcLabel}
%http://www.mobilepaymentstoday.com/blogs/ble-vs-nfc-the-future-of-mobile-consumer-engagement-now-infographic/


\subsection{Bluetooth}

